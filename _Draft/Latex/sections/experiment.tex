\renewcommand\thetable{\arabic{chapter}-\arabic{table}}
%\renewcommand\thefigure{\arabic{chapter}-\arabic{figure}} 
\chapter{實驗結果與分析}
\label{cha:experiment}

為了探討前述方法是否符合需求目標,將進行以下實驗:~\ref{sec:experiment-xxx} 小節中,針對所挑選的房間採用不同權重的適應性函數,會如何影響房間內的遊戲物件之配置結果。~\ref{sec:experiment-yyy} 小節中,觀察染色體中的基因數量對於演化過程之影響。~\ref{sec:experiment-zzzz} 小節,進行評估不同的交配策略下,何者優劣優勢之情形。

\section{實驗定義}
\label{sec:experiment-definition}

在任務語法階段時,我們將非合法 ... (編輯中)

地圖片段演化階段時,每一次世代的演化過程,有 80\% 機率父母代間會進行兩點交配 (two-point crossover);10\% 機率衍生子代會進行突變,染色體個體中有 5\% 至 20\% 的基因數量會轉換成其它的物件種類。

\section{資料收集}
\label{sec:experiment-datacollection}

在資料收集階段中,將收集~\ref{sec:method-segments} 基因演算法於各實驗、世代與其個體(單一染色體)中基因類型的得分狀況。

\begin{table}[ht]
  \centering
  \caption{原始資料之欄位}
  \label{tbl:structure-of-rawdata}
  \bigskip
  \begin{tabular}{| l | l | l | l | l | l | l | l |}
    \hline
    實驗編號 & 世代編號 & 染色體編號 & 指標 & 得分 & 座標 & 類型 & 房間 \\
    1 & 1 & 1 & Block     & 0 & (12.0, 1.0, 0.0)  & Empty & Room A \\
    1 & 1 & 1 & Intercept & 0 & (12.0, 1.0, 0.0)  & Empty & Room A \\
    1 & 1 & 1 & Patrol    & 0 & (12.0, 1.0, 0.0)  & Empty & Room A \\
    1 & 1 & 1 & Guard     & 0 & (12.0, 1.0, 0.0)  & Empty & Room A \\
    1 & 1 & 1 & Support   & 0 & (12.0, 1.0, 0.0)  & Empty & Room A \\
    1 & 1 & 1 & Block     & 0 & (24.0, 1.0, 12.0) & Empty & Room A \\
    1 & 1 & 1 & Intercept & 0 & (24.0, 1.0, 12.0) & Empty & Room A \\
    \hline
  \end{tabular}
\end{table}

\section{各世代演化狀態}
\label{sec:experiment-evolutions}

由於各點基因,其中的空格、敵人兩種類型,於每一遊玩特徵的指標都具有高度意義與相關性。

\section{演化結果與其品質}
\label{sec:experiment-xxx}

content here.

\section{房型規模之比較}
\label{sec:experiment-yyy}

房型的大小較有可能直接影響和可行走瓦磚之數量。本階段的實驗中,我們提取~\ref{sec:experiment-datacollection} 節的資料,將空白、敵人兩種類型的數量關係繪製成熱圖進行觀察。這是因為各項適應性函數在設計時,多以「敵人」與其餘敵人、其它遊戲物件或玩家動線為考量參考,因而推估二者間勢必存在者某些關係。

\section{評估交配策略優勢}
\label{sec:experiment-zzzz}

content here.

