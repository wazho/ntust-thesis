\chapter{研究方法}
\label{cha:methodology}

這本篇論文中,我們仍保持 Joris Dormans 與 Antonios Liapis 為求遊戲設計過程抽象化與高階化的訴求。我們將「Mission/Space框架」與「Multi-segment演化」兩種關卡生成方法結合,保留了前者追求的遊戲進程之順序性,後者帶來穩定且多樣化的遊戲內容,希冀藉此提升整體遊戲體驗、相輔相成。

圖 OOOOO 為系統的整體流程圖。遊戲設計師能夠透過巨觀的觀點來構築遊戲體驗流,將遊玩特徵拆分成多項規則,利用生成語法及改寫系統生成出任務圖。依照任務圖中對應的終端節點 (terminal nodes) 轉換為事先建構完成的房型空間,並得到尚未包含遊戲物件的遊戲地圖。接下來,針對動作冒險遊戲我們提出了數項評估遊戲性的適應性函數,採用基因演算法的過程,令各房間遵守適應性函數的限制得到符合訴求的最佳解。

\section{任務語法}



\section{房型建構}

Joris Dormans 於文獻中提到為二維空間的範例,我們的實驗環境以三維空間為主。在空間語法中我們將直接構築遊戲的基礎房型,但不設置怪物、寶箱或陷阱足以直接影響遊戲性的遊戲物件,如圖 OOOOO。此外,我們希望空間中的遊戲物件能夠有意義的自動化配置,即在設計空間語法的流程中,忽略絕大部分的遊戲物件配置,直到++3.3++節提出之方法達成。

我們對於空間語法做了修改以利實驗環境建置,在一個關卡 (level) 中包含數個房間容器 (volumes),每一房間由不定數量的房間塊 (chucks) 組成,且房間塊固定以 9x9x9 個正立方體體素 (voxels) 所構成。為了要從已生成完畢的任務圖再衍生出分支的遊戲空間,在我們創建的房間容器會對應一任務語法之字母表當中一的終端符 (terminal symbols),這樣的關係稱作為建造指示(building instruction);若在改寫規則運行中,且有多項規則同時符合替換的條件時,系統會基於它們的關聯權重 (relative weight) 隨機挑選一個規則。


\section{地圖片段}

延續上一小節的關卡結構,房間視為染色體 (chromosome);房間中能夠放置遊戲物件的各點座標視為基因 (genes),而基因的類型有空磚、怪物磚、寶箱磚與陷阱磚等;同一個房間會擁有多種不同遊戲物件配置情況的染色體,這些染色體的集合便是族群 (population)。第一步驟,產生初始父母代族群時,讓全部的基因先預設為空磚,並使各染色體先行隨機突變;第二步驟,透過適應性函數計算各染色體的適應值 (fitnesses),完整的適應性函數在++3.3.1++小節中說明;第三步驟將會從族群中挑選最優異的兩個父母染色體,高機率進行交配,若無進行交配將會將子代沿用父母代的基因,採用的交配方法在++3.3.2++小節中說明;第四步驟有低機率讓衍生的子代進行突變,不同的突變方式於++3.3.3++小節中說明;第五步驟以新的衍生子代取代舊有的父母代族群;第六步驟會檢查是否達到終止條件,若尚未滿足終止條件,便會回到第二步驟,直到輸出最佳解。

