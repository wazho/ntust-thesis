\chapter{研究方法}
\label{cha:methodology}

這本篇論文中,我們仍保持 Joris Dormans 與 Antonios Liapis 為求遊戲設計過程抽象化與高階化的訴求。我們將「Mission/Space框架」與「Multi-segment演化」兩種關卡生成方法結合,保留了前者追求的遊戲進程之順序性,後者帶來穩定且多樣化的遊戲內容,希冀藉此提升整體遊戲體驗、相輔相成。

圖 OOOOO 為系統的整體流程圖。遊戲設計師能夠透過巨觀的觀點來構築遊戲體驗流,將遊玩特徵拆分成多項規則,利用生成語法及改寫系統生成出任務圖。依照任務圖中對應的終端節點 (terminal nodes) 轉換為事先建構完成的房型空間,並得到尚未包含遊戲物件的遊戲地圖。接下來,針對動作冒險遊戲我們提出了數項評估遊戲性的適應性函數,採用基因演算法的過程,令各房間遵守適應性函數的限制得到符合訴求的最佳解。

\section{任務語法}



\section{房型建構}

Joris Dormans 於文獻中提到為二維空間的範例,我們的實驗環境以三維空間為主。在空間語法中我們將直接構築遊戲的基礎房型,但不設置怪物、寶箱或陷阱足以直接影響遊戲性的遊戲物件,如圖 OOOOO。此外,我們希望空間中的遊戲物件能夠有意義的自動化配置,即在設計空間語法的流程中,忽略絕大部分的遊戲物件配置,直到++3.3++節提出之方法達成。

我們對於空間語法做了修改以利實驗環境建置,在一個關卡 (level) 中包含數個房間容器 (volumes),每一房間由不定數量的房間塊 (chucks) 組成,且房間塊固定以 9x9x9 個正立方體體素 (voxels) 所構成。為了要從已生成完畢的任務圖再衍生出分支的遊戲空間,在我們創建的房間容器會對應一任務語法之字母表當中一的終端符 (terminal symbols),這樣的關係稱作為建造指示(building instruction);若在改寫規則運行中,且有多項規則同時符合替換的條件時,系統會基於它們的關聯權重 (relative weight) 隨機挑選一個規則。


\section{地圖片段}

延續上一小節的關卡結構,房間視為染色體 (chromosome);房間中能夠放置遊戲物件的各點座標視為基因 (genes),而基因的類型有空磚、怪物磚、寶箱磚與陷阱磚等;同一個房間會擁有多種不同遊戲物件配置情況的染色體,這些染色體的集合便是族群 (population)。第一步驟,產生初始父母代族群時,讓全部的基因先預設為空磚,並使各染色體先行隨機突變;第二步驟,透過適應性函數計算各染色體的適應值 (fitnesses),完整的適應性函數在++3.3.1++小節中說明;第三步驟將會從族群中挑選最優異的兩個父母染色體,高機率進行交配,若無進行交配將會將子代沿用父母代的基因,採用的交配方法在++3.3.2++小節中說明;第四步驟有低機率讓衍生的子代進行突變,不同的突變方式於++3.3.3++小節中說明;第五步驟以新的衍生子代取代舊有的父母代族群;第六步驟會檢查是否達到終止條件,若尚未滿足終止條件,便會回到第二步驟,直到輸出最佳解。

\subsection{各項適應性函數}

動作冒險遊戲 (A-AVG)、動作角色扮演遊戲 (A-RPG) 等類型遊戲,多可見一些制式化遊戲物件的搭配組合,我們嘗試汲取出多項遊玩特徵並參數化公式,作為評估關卡品質的指標之一。在前處理時,我們使用 A-Star 演算法尋找入口至多個出口的最短路徑,凡經過的座標稱作為空間動線 ($MP$) 之一,並將其權重值 ($mp$) 增加一,空間動線為多項指標關鍵性的參考依據。


\subsubsection{死角點 (Neglected)}

由於房與房之間的牆壁阻隔,使得敵人能夠埋伏於入口附近之死角處,出奇不意地對玩家展開攻擊。為了體現出這種現象,我們將敵人 ($E_{i}$) 與主要動線上各點 ($MP_{i}$),兩端點連線之對角線所構成的立方體,立方體所涵蓋各座標點 ($N_{j}$) 至該對角線的距離為 $d_{k}$,隨著距離增加影響程度會衰減;$vis_{k}$ 為該點的可視情形,若有不可視的座標存在便會提高適應值。隨著動線的順序演進,影響程度逐漸衰減。

\begin{equation}
Under construction
\end{equation}

\subsubsection{阻攔點 (Block)}

敵人會專注於阻擋玩家繼續前進,迫使玩家與其發生衝突。會配置於動線之上,$mp_{i}$ 為空間動線權重;$e_j$ 為該敵人於空間之動線權重,倘敵人並未落在動線上,則該項為 0。

\begin{equation}
f_{blk}=\frac{\sum_{j=1}^{M} e_{j}}{\sum_{i=1}^{N} mp_{i}}
\end{equation}

\subsubsection{攔截點 (Intercept)}

與阻攔點近似,但敵人會被配置於動線附近非動線上,以快速追擊玩家為目的。各敵人 ($E_{i}$) 越接近空間動線各點 ($MP_{j}$) 時影響愈大,且動線權重 ($mp_{j}$) 亦會影響加權程度。

\begin{equation}
f_{itc}=\frac{1}{N \times M} \sum_{i=1}^{N} \sum_{j=1}^{M} \Big( \frac{1}{dist(E_{i}, MP_{j})} \times mp_{j} \Big), E_{i} \neq MP_{j}
\end{equation}

\subsubsection{巡邏點 (Patrol)}

確保各敵人擁有足夠的空間能夠進行移動。將計算敵人 ($E_{i}$) 與指定半徑 ($R$) 內的座標數量 ($P_j$) 總值,當中並不包含不可通行的牆壁。於本次實驗中,我們將採用 $R=3$ 作為實驗範例,該數值可由遊戲設計師決定。

\begin{equation}
f_{ptl}=\sum_{i=1}^{N} \sum_{j=1}^{M} count\big(E_{i}, P_{j}\big), dist\big(E_{i}, P_{j}\big) \leq R, P_{j} \notin \{ wall \}
\end{equation}

\subsubsection{守衛點 (Guard)}

為體現出敵人會保衛寶箱 ($T$) 與出口 ($E$) 的現象,計算敵人 ($E_{i}$) 與關鍵性較高遊戲物件 ($O_{j}$) 之間的距離,倘若距離愈近則帶來的影響力愈大。

\begin{equation}
f_{grd}=\frac{1}{N \times M} \sum_{i=1}^{N} \sum_{j=1}^{M} \frac{1}{dist\big(E_{i}, O_{j}\big)}, O_{j} \in \{ Treasure, Exit \}
\end{equation}

\subsubsection{至高點 (Dominated)}

當玩家可能所在動線上之位置 ($MP_{j}$) 與敵人的位置 ($E_{i}$) 具有高低差時,敵人便適合採取遠程攻擊;為了提供玩家思考對付遠程敵人的緩衝時間,將敵人配置於動線末端附近是較好的選擇,$j$ 隨著動線的順序演進,影響程度逐漸增幅。

\begin{equation}
f_{dom}=\sum_{i=1}^{N} \sum_{j=1}^{M} \Big( \frac{1}{dist\big(E_{i}, MP_{j}\big)} \times mp_{j} \times j \times high(E_{i}, MP_{j}) \Big)
\end{equation}

\subsubsection{支援點 (Support)}

敵人 ($E_{i}$, $E_{j}$) 之間擁有一定程度的護援關係,當敵人彼此的距離愈低其影響程度越大,同時該敵人 ($E_{i}$) 必須遠離動線 ($MP_{k}$)。

\begin{equation}
f_{sup}=\frac{1}{N} \sum_{i=1}^{N} \bigg( \frac{1}{N} \sum_{\substack{j=1 \\ j \neq i}}^{N} \frac{1}{dist(E_{i}, E_{j})} + \frac{1}{M} \sum_{k=1}^{M} \frac{1}{dist(E_{i}, MP_{k})} \bigg)
\end{equation}

\subsection{不同交配方式}

\subsection{不同的突變方式}

\subsection{終止條件的設立}
