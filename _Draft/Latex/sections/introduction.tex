\chapter{緒論}
\label{cha:intro}

程序化內容自動生成 (Procedural Content Generation) 在過去就廣泛被應用於遊戲設計領域,其主要目的為增加遊戲內容的隨機性與多樣性。在本文中,我們針對遊戲過程中的遊玩特徵 (gameplay patterns) 進行抽象化,使用程序化生成技術產生帶有意義遊戲關卡內容,藉此消彌或降低因隨機性所產生的不穩定要素,以改善並豐富遊戲體驗。

我們將遊戲關卡的構成劃分為任務 (Missions) 與空間 (Space) 兩種結構後,空間會依照任務結構進行有意義的轉換,接著依照遊玩特徵定義基因演算法 (Genetic Algorithms) 的演化依據。讓玩家在進行遊戲時能夠遵循關卡設計師的劇情脈絡外,亦能夠體驗到有意義且多樣化的遊戲關卡內容。

\section{研究背景與動機}

content here.

\subsection{迷宮探索遊戲 (dungeon crawl) 類型介紹}

content here.

\subsubsection{迷宮探索遊戲的歷史}

content here.


\section{研究目的與研究問題}

由 Joris Dormans 提出的自動關卡設計方法中,藉由任務語法精準控制玩家遊玩的進程,而遊戲物件的佈局由空間語法掌控,卻無法控制其品質優劣。由 Antonios Liapis 提出的戰略型遊戲抽象化地圖生成方法中,對於遊戲物件的佈局提供了評定品質的量測方式。

\begin{table}[!htb]
  \centering
  \caption{Joris Dormans 與 Antonios Liapis 提出之方法,進行綜觀比較}
  \label{tbl:compare-the-method-form-jd-and-al}
  \bigskip
  \begin{tabular}{ | c | c | c | }
    \hline
    方法框架                               & 任務進程 & 內容評估     \\\hline
    Mission/Space 框架(Joris Dormans)    & 精準控制 & 無法評估優劣 \\\hline
    Sentient Sketchbook(Antonios Liapis) & 未提及   & 提供優劣評估 \\\hline
  \end{tabular}
\end{table}

\subsection{研究目的}

我們將參考上一段落所提及的兩種方法框架,針對遊戲關卡的意義性與關卡品質進行研究。研究目的下列所述:

\begin{enumerate}
  \item 設計並提出能夠生成帶有意義的 3D 動作遊戲的關卡自動生成演算法。
  \item 以基因演算法輔助該自動生成演算法,進行自我品質評估與改良。
\end{enumerate}

\subsection{研究問題}

根據上述之研究目的,本研究提出以下研究問題:

\begin{enumerate}
  \item 存在多種不同面向的適應性函數,要如何進行有效的數值標準化?
  \item 人為調整相關基因演算法演化參數時,對於關卡生成會有什麼影響?
\end{enumerate}

\subsection{預計研究貢獻}

content here.

\section{本論文之章節結構}

content here.

